\capitulo{1}{Introducción}

Una forma de poder determinar los conocimientos de cualquiera es con la realización de un test, de esta forma se vinculan las \textbf{Soft Skills} también denominadas como habilidades blandas las cuales nos muestran cómo se comporta en diferentes escenarios el ser humano.
A partir de los valores obtenidos se puede realizar un estudio de cómo se comportara en diferentes escenarios y así tener una idea de que rol ejercerá en un equipo o como actuara ante imprevistos.
Una vez dicho esto RiskReal es una página web desde la cual se pueden realizar diferentes cuestionarios o tests para comprobar las Soft Skills que tiene el usuario.
Lo que se pretende con este proyecto es la implementación de una plataforma en la cual se pueden tener diferentes test y cuestionarios independientes para que los puedan realizar los diferentes usuarios registrados en la plataforma como de forma anónima.
De otra manera los usuarios que tengan el rol de administrador tienen acceso al listado de los usuarios registrados y al listado de los resultados de los test, teniendo la opción de descargar los datos en un archivo “.csv” para tratar los datos.


\section{Estructura de la Memoria}
\begin{enumerate}
    \item \textbf{Introducción.} Breve descripción sobre el proyecto.
    \item \textbf{Objetivos del proyecto.} Diferentes objetivos que se persiguen con el proyecto
    \item \textbf{Conceptos teóricos.} Explicación de los conceptos y términos en relación con el proyecto
    \item \textbf{Técnicas y herramientas.} Listado de herramientas y bibliotecas utilizadas con una breve explicación.
    \item \textbf{Aspectos relevantes.} Listado de los puntos más interesantes e importantes que han ido surgiendo durante el desarrollo del proyecto.
    \item \textbf{Trabajos relacionados.} Exposición de trabajos y plataformas relacionadas con RiskReal.
    \item \textbf{Conclusiones y Líneas de trabajo futuras.} Conclusiones respecto a los diferentes apartados del proyecto y diferentes mejoras que se pueden implementar al proyecto.
\end{enumerate}

\section{Estructura de los Anexos}

\begin{enumerate}[A.]
    \item \textbf{Plan del proyecto}: Plan de proyecto software que estudia la planificación y viabilidad del proyecto. 
    \item \textbf{Especificación de Requisitos}: Catálogo de los requisitos funcionales y no funcionales del proyecto.
    \item \textbf{Especificación de Diseño}: Diseño general de los distintos apartado del sistema así como su justificación.
    \item \textbf{Manual de Programador}: Manual con toda la información importante que permita a otro programado retomar el proyecto rápidamente.
    \item \textbf{Manual de Usuario}: Manual de usuario que explora todas las características de la plataforma web. 
    \item \textbf{Anexo de sostenibilización curricular}: Sostenibilidad del proyecto respecto a las ODS.
\end{enumerate}
