\capitulo{6}{Trabajos relacionados}

El caso claro sobre un trabajo relacionado es el RiskReal original ~\cite{RiskReal} el cual realizo la universidad de Burgos en colaboración con la Unión Europea  por lo que vamos a ver las diferentes aportaciones que nos ofrecen las dos versiones.

\noindent
\begin{table}[]
    \centering
    \begin{tabularx}{\textwidth}{|X|c|c|}
        \hline
        Funcionalidad & Original & Mi proyecto \\ \hline
        Registro & Sí & Sí \\
        Login & Sí & Sí\\
        Opción de Invitado & Sí & Sí\\
        Cambiar la selección & Solo cuestionario & Ambos \\
        Opcion de anterior y siguiente en los cuestionarios/tests & No & Sí\\
        Indicador pregunta & Solo cuestionario & Ambos \\
        Salir del cuestionario/test & Sí & No \\
        Cambio de contraseña & Sí & Sí\\
        Rol de administrador & ND & Sí\\
        Consulta de los usuarios & ND & Sí\\
        Consulta de los resultados & ND & Sí\\
        Opción de descarga de los resultados & ND & Sí\\
        Disponibilidad a todos los cuestionarios/tests & No & Sí\\
        Añadir nuevos cuestionarios/tests & No & Sí\\
        Tiempo entre las transacciones & Medio & Bajo \\ \hline
    \end{tabularx}
    \caption{Tabla con diferencias entre RiskReals}
    \label{tab:my_label}
\end{table}

Hay varios aspectos que no puedo comprobar respecto al no ser un administrador, lo cual lo tomare como una clara ventaja respecto a la original, a parte la funcionalidad que ofrecen los diferentes cuestionarios y test en mi caso se mantienen pero en el caso del original no tienen las mismas funcionalidades.
Hay algunas disponibilidades que no se comparten como poder salir en mitad de la realización de un cuestionario/test lo cual es una línea de trabajo futura.
En cuanto a la disponibilidad de todos los cuestionarios/tests me refiero a que en mi caso el botón desplegable te muestra todos aquellos respectivamente a su categoría, es decir en el original solo se puede hacer un cuestionario y un test.