\apendice{Especificación de Requisitos}

\section{Introducción}

En este anexo se describirán diferentes aspectos acerca de los requisitos del proyecto.

\section{Objetivos generales}
\begin{enumerate}
    \item Desarrollo de dos funciones independientes capaces de desglosar un json, con el fin de obtener por una parte el número total de preguntas que contiene y por otra parte todos los elementos necesarios para mostrar el test completo, así el usuario tiene la capacidad de realizar cualquier tipo de test o cuestionario disponible en la página.
    \item Crear una aplicación que responda al usuario correctamente asegurándonos de realizar lo marcado y que el propio usuario sienta el cambio o su posible cambio.  
    \item Posibilidad de poder descargar todos los datos del test para los administradores, con el fin de poder hacer estudios a parte o sacar valores como la moda para poder realizar tests o cuestionarios acorde a lo solicitado.
    \item Garantizar que la aplicación tenga un rendimiento estable y que sea sostenible, además de ser “responsive” ante todo tipos de dispositivos móviles y tabletas a parte del navegador web de un ordenador.
\end{enumerate}
\section{Catálogo de requisitos}
\subsection{Requisitos Funcionales}
Son aquellos requisitos cuya característica se espera en el proyecto.
\begin{itemize}
    \item \textbf{RF-1 Tratamiento de usuarios:} Se requiere que el sistema sea capaz de tratar a los usuarios
    \begin{itemize}
        \item \textbf{RF-1.1 Registro:} El sistema debe ser capaz de crear nuevos usuarios.
        \item \textbf{RF-1.2 Inicio Sesión:} El sistema debe ser capaz de crear una sesión  para el usuario.
        \item \textbf{RF-1.3 Cerrar Sesión:} El sistema debe permitir cerrar la sesión  del usuario.
        \item \textbf{RF-1.4 Cambio de contraseña:} El sistema puede cambiar la contraseña del usuario que lo desee.
        \item \textbf{RF-1.5 Participación como Invitado:} El sistema debe ser capaz de permitir a un usuario que no esté registrado interactuar con la aplicación.
    \end{itemize}
    \item \textbf{RF-2 Tratamiento de pruebas:} Se requiere que el sistema soporte las diferentes formas de evaluarse.
    \begin{itemize}
        \item \textbf{RF-2.1 Realización de tests/cuestionarios:} El sistema debe dejar realizar tanto a un usuario o invitado a realizar cualquier tipo de test y cuestionario.
        \item \textbf{RF-2.2 Navegación en los test/cuestionarios:} El sistema debe permitir al usuario poder moverse entre las cuestiones o salirse de el en caso de que no quiera continuarlo.
        \item \textbf{RF-2.3 Calculo y almacenamiento de los resultados:} El sistema debe mostrar la calificación correspondiente y almacenarla en la BBDD.
    \end{itemize}
    \item \textbf{RF-3 Visualización de datos:} Se requiere que el sistema tenga una forma de poder ver los datos generados por la aplicación.
    \begin{itemize}
        \item \textbf{RF-3.1 Visualización de los usuarios:} El sistema debe dejar ver a los administradores todos los usuarios que estén  registrados en la plataforma.
        \item \textbf{RF-3.2 Visualización de los resultados:} El sistema debe dejar ver a los administradores todos los resultados de los test y cuestionarios que estén  en la plataforma.
    \end{itemize}
    \item \textbf{RF-4 Exportación de datos:} El sistema debe dejar a los administrador poder exportar todos los resultados en un archivo "resultados.csv"
\end{itemize}
\subsection{Requisitos no Funcionales}
Son aquellos requisitos que definen como se debe comportar el programa en cualquier momento.
\begin{itemize}
    \item \textbf{RNF-1 Seguridad:} Se requiere un mínimo de seguridad con el tratamiento de datos sensibles.
    \begin{itemize}
        \item \textbf{RNF-1.1 Datos hash:} Las contraseñas y las respuestas a la pregunta secretas se guardan como valores hash.
        \item \textbf{RNF-1.2 Claves secretas:} Se dispone de una secret\_key para la gestión de sesiones
    \end{itemize}
    \item \textbf{RNF-2 Rendimiento:} La aplicación debe mantener un rendimiento para evitar tiempos largos de carga.
    \item \textbf{RNF-3 Escalabilidad:} La aplicación es capaz de manejar múltiples usuarios simultaneamente incluyendo usuarios repetidos e invitados.
    \item \textbf{RNF-4 Usabilidad:} La aplicación es comprensible para los usuarios.
    \begin{itemize}
        \item \textbf{RNF-4.1 Navegación intuitiva:} La interfaz del usuario es intuitiva y permite una fácil navegación entre las funcionalidades
        \item \textbf{RNF-4.2 Tratamiento de errores:} Siempre que ocurra un error se mostrara por pantalla, para que en caso de introducir mal los datos el usuario lo sepa o se tenga que poner en contazto con un administrador.
    \end{itemize}
    \item \textbf{RNF-5 Compatibilidad:} La aplicación debe ser compatible en diferentes plataformas
    \item \textbf{RNF-6 Mantenimiento y Extensibilidad:} Es posible mantener la aplicación y añadir nuebas funcionalidades.
    \item \textbf{RNF-7 Configuración y Despliegue:} La aplicación puede ser desplegada en diferentes entornos o cualquier servidor compatible con Flask.
\end{itemize}
\section{Especificación de requisitos}
Este apartado esta reservado para explicar los casos de uso del proyecto
\subsection{Actores}
Se disponen de los diferentes actores en el sistema:
\begin{itemize}
    \item \textbf{Usuarios:} todos los usuarios que aceden a la aplicación que a su vez se desglosa en:
    \begin{itemize}
        \item \textbf{Usuario Registrado:} Todos aquellos que esten registrados en la pltaforma.
        \item \textbf{Administrador:} Encargados de administrar la aplicación que a su vez son usuarios registrados.
        \item \textbf{Invitado:} Todos aquellos que quieran evaluarse sin la necesidad de registrarse.
    \end{itemize}
    \item \textbf{Sistema:} Se encarga de toda la lógica y procesar la informacíon recibida por los diferentes usuarios.
\end{itemize}

\subsection{Diagrama de Casos de Uso}
El diagrama de casos de uso que contiene los diferentes actores.

\imagen{DiagramaCasosDeUso}{Diagrama de Casos de Uso}

\subsection{Casos de Uso}
Los diferentes casos de uso correspondientes al proyecto.

\begin{table}[p]
	\centering
	\begin{tabularx}{\linewidth}{ p{0.21\columnwidth} p{0.71\columnwidth} }
		\toprule
		\textbf{CU-1}    & \textbf{Registro de Usuario}\\
		\toprule
		\textbf{Versión}              & 1.0    \\
		\textbf{Autor}                & Marcos Ubierna Fernández \\
		\textbf{Requisitos asociados} & RF-1.1 \\
		\textbf{Descripción}          & El usuario puede registrarse proporcionando su información personal. \\
		\textbf{Precondición}         & El usuario no debe estar registrado previamente. \\
		\textbf{Acciones}             &
		\begin{enumerate}
			\def\labelenumi{\arabic{enumi}.}
			\tightlist
			\item El usuario navega a la página de registro.
			\item El usuario ingresa su correo electrónico, nombre, apellido, género, edad, rol, una pregunta secreta y una contraseña.
                \item El sistema verifica que el correo electrónico no esté registrado.
                \item El sistema almacena la información del usuario en la base de datos.
                \item El sistema muestra un mensaje de confirmación de registro exitoso.
		\end{enumerate}\\
		\textbf{Postcondición}        & El usuario está registrado y puede iniciar sesión. \\
		\textbf{Excepciones}          & El correo electrónico ya está registrado: el sistema muestra un mensaje de error. \\
		\textbf{Importancia}          & Alta \\
		\bottomrule
	\end{tabularx}
	\caption{CU-1  Registro de Usuario.}
\end{table}

\begin{table}[p]
	\centering
	\begin{tabularx}{\linewidth}{ p{0.21\columnwidth} p{0.71\columnwidth} }
		\toprule
		\textbf{CU-2}    & \textbf{Inicio de Sesión}\\
		\toprule
		\textbf{Versión}              & 1.0    \\
		\textbf{Autor}                & Marcos Ubierna Fernández \\
		\textbf{Requisitos asociados} & RF-1.2, RF-1.3 \\
		\textbf{Descripción}          & El usuario puede iniciar sesión con su correo electrónico y contraseña. \\
		\textbf{Precondición}         & El usuario debe estar registrado. \\
		\textbf{Acciones}             &
		\begin{enumerate}
			\def\labelenumi{\arabic{enumi}.}
			\tightlist
		      \item El usuario navega a la página de inicio de sesión.
		      \item El usuario ingresa su correo electrónico y contraseña.
                \item El sistema verifica las credenciales.
                \item Si las credenciales son correctas, el sistema crea una sesión para el usuario.
                \item El sistema redirige al usuario a la página principal.
		\end{enumerate}\\
		\textbf{Postcondición}        & El usuario está autenticado y puede acceder a las funcionalidades del sistema. \\
		\textbf{Excepciones}          & Credenciales incorrectas: el sistema muestra un mensaje de error. \\
		\textbf{Importancia}          & Alta \\
		\bottomrule
	\end{tabularx}
	\caption{CU-2 Inicio de Sesión.}
\end{table}

\begin{table}[p]
	\centering
	\begin{tabularx}{\linewidth}{ p{0.21\columnwidth} p{0.71\columnwidth} }
		\toprule
		\textbf{CU-3}    & \textbf{Realización de Tests/Cuestionarios}\\
		\toprule
		\textbf{Versión}              & 1.0    \\
		\textbf{Autor}                & Marcos Ubierna Fernándes \\
		\textbf{Requisitos asociados} & RF-2.1, RF-2.2 \\
		\textbf{Descripción}          &Los usuarios pueden realizar tests y cuestionarios disponibles. \\
		\textbf{Precondición}         & El usuario debe estar registrado o actuar como invitado. \\
		\textbf{Acciones}             &
		\begin{enumerate}
			\def\labelenumi{\arabic{enumi}.}
			\tightlist
			\item El usuario navega a la página de tests/cuestionarios.
			\item El usuario selecciona un test/cuestionario.
                \item El sistema presenta las preguntas del test/cuestionario.
                \item El usuario responde las preguntas.
                \item El usuario envía el test/cuestionario completado.
                \item El sistema calcula y almacena los resultados.
                \item El sistema muestra los resultados al usuario.
		\end{enumerate}\\
		\textbf{Postcondición}        &  Los resultados del test/cuestionario están almacenados en la base de datos. \\
		\textbf{Excepciones}          & Fallo en el almacenamiento de los resultados: el sistema muestra un mensaje de error. \\
		\textbf{Importancia}          & Alta \\
		\bottomrule
	\end{tabularx}
	\caption{CU-3 Realización de Tests/Cuestionarios.}
\end{table}

\begin{table}[p]
	\centering
	\begin{tabularx}{\linewidth}{ p{0.21\columnwidth} p{0.71\columnwidth} }
		\toprule
		\textbf{CU-4}    & \textbf{Visualización de Usuarios Registrados}\\
		\toprule
		\textbf{Versión}              & 1.0    \\
		\textbf{Autor}                & Marcos Ubierna Fernández \\
		\textbf{Requisitos asociados} & RF-3.1 \\
		\textbf{Descripción}          & Los administradores pueden ver todos los usuarios registrados en la plataforma. \\
		\textbf{Precondición}         & El usuario debe tener rol de administrador. \\
		\textbf{Acciones}             &
		\begin{enumerate}
			\def\labelenumi{\arabic{enumi}.}
			\tightlist
			\item El administrador navega a la página de administración.
			\item El administrador selecciona la opción para ver usuarios registrados.
                \item El sistema muestra una lista paginada de usuarios registrados.
		\end{enumerate}\\
		\textbf{Postcondición}        & El administrador puede ver los detalles de los usuarios registrados. \\
		\textbf{Excepciones}          & Error en la carga de usuarios: el sistema muestra un mensaje de error. \\
		\textbf{Importancia}          & Media \\
		\bottomrule
	\end{tabularx}
	\caption{CU-4 Visualización de Usuarios Registrados.}
\end{table}

\begin{table}[p]
	\centering
	\begin{tabularx}{\linewidth}{ p{0.21\columnwidth} p{0.71\columnwidth} }
		\toprule
		\textbf{CU-5}    & \textbf{Visualización de los resultados}\\
		\toprule
		\textbf{Versión}              & 1.0    \\
		\textbf{Autor}                & Marcos Ubierna Fernández \\
		\textbf{Requisitos asociados} & RF-3.2 \\
		\textbf{Descripción}          & Los administradores pueden ver todos los resultados de los test/cuestionarios de la plataforma. \\
		\textbf{Precondición}         & El usuario debe tener rol de administrador. \\
		\textbf{Acciones}             &
		\begin{enumerate}
			\def\labelenumi{\arabic{enumi}.}
			\tightlist
			\item El administrador navega a la página de administración.
			\item El administrador selecciona la opción para ver los resultados.
                \item El sistema muestra una lista paginada de los resultados de los test/cuestionarios.
		\end{enumerate}\\
		\textbf{Postcondición}        & El administrador puede ver los detalles de los test/cuestionarios realizados. \\
		\textbf{Excepciones}          & Error en la carga de resultados: el sistema muestra un mensaje de error. \\
		\textbf{Importancia}          & Media \\
		\bottomrule
	\end{tabularx}
	\caption{CU-5 Visualización de los resultados.}
\end{table}

\begin{table}[p]
	\centering
	\begin{tabularx}{\linewidth}{ p{0.21\columnwidth} p{0.71\columnwidth} }
		\toprule
		\textbf{CU-6}    & \textbf{Exportación de Resultados}\\
		\toprule
		\textbf{Versión}              & 1.0    \\
		\textbf{Autor}                & Alumno \\
		\textbf{Requisitos asociados} & RF-4 \\
		\textbf{Descripción}          & Los administradores pueden exportar los resultados de tests/cuestionarios en un archivo CSV. \\
		\textbf{Precondición}         & El usuario debe tener rol de administrador. \\
		\textbf{Acciones}             &
		\begin{enumerate}
			\def\labelenumi{\arabic{enumi}.}
			\tightlist
			\item El administrador navega a la página de administración.
			\item El administrador selecciona la opción para ver los resultados.
                \item El administrador selecciona la opción de descargar csv.
		\end{enumerate}\\
		\textbf{Postcondición}        & El administrador obtiene un archivo CSV con los resultados. \\
		\textbf{Excepciones}          & Error en la generación del archivo CSV: el sistema muestra un mensaje de error \\
		\textbf{Importancia}          & Media \\
		\bottomrule
	\end{tabularx}
	\caption{CU-6 Exportación de Resultados.}
\end{table}