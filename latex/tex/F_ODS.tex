\apendice{Anexo de sostenibilización curricular}

\section{Introducción}
La sostenibilidad ~\cite{wiki:Sostenibilidad} se ha convertido en un aspecto cada vez más importante el cual se basa en  buscar un enfoque integral para el desarrollo con el fin de equilibrar las necesidades económicas, sociales y ambientales para garantizar el bienestar presente sin comprometer el de las futuras generaciones.

Según la Agenda 2030 existen un total de 17 objetivos de desarrollo sostenible los cuales son formados por ~\cite{ODS}:
\imagen{ODS}{Objetivos de Desarrollo Sostenible}
\begin{enumerate}
    \item \textbf{Fin de la pobreza:} erradicar la pobreza en todas sus formas y en todo el mundo.
    \item \textbf{Hambre cero:} eliminar el hambre y asegurar el acceso de todas las personas, en particular los pobres y personas en situaciones vulnerables, a alimentos nutritivos y suficientes durante todo el año.
    \item \textbf{Salud y bienestar:} garantizar una vida sana y promover el bienestar para todos en todas las edades.
    \item \textbf{Educación de calidad:} garantizar una educación inclusiva, equitativa y de calidad, y promover oportunidades de aprendizaje durante toda la vida para todos.
    \item \textbf{Igualdad de género:} lograr la igualdad de género y empoderar a todas las mujeres y niñas.
    \item \textbf{Agua limpia y saneamiento:} garantizar la disponibilidad y la gestión sostenible del agua y el saneamiento para todos.
    \item \textbf{Energía asequible y no contaminante:} garantizar el acceso a una energía asequible, fiable, sostenible y moderna para todos.
    \item \textbf{Trabajo decente y crecimiento económico:} promover el crecimiento económico sostenido, inclusivo y sostenible, el empleo pleno y productivo, y el trabajo decente para todos.
    \item \textbf{Industria, innovación e infraestructura:} construir infraestructuras resilientes, promover la industrialización inclusiva y sostenible, y fomentar la innovación.
    \item \textbf{Reducción de las desigualdades:} reducir la desigualdad en y entre los países.
    \item \textbf{Ciudades y comunidades sostenibles:} lograr que las ciudades y los asentamientos humanos sean inclusivos, seguros, resilientes y sostenibles.
    \item \textbf{Producción y consumo responsable:} garantizar modalidades de consumo y producción sostenibles.
    \item \textbf{Acción por el clima:} adoptar medidas urgentes para combatir el cambio climático y sus efectos.
    \item \textbf{Vida submarina:} conservar y utilizar de manera sostenible los océanos, los mares y los recursos marinos.
    \item \textbf{Vida de ecosistemas terrestres:} gestionar de manera sostenible los bosques, combatir la desertificación, detener e invertir la degradación de las tierras y detener la pérdida de biodiversidad.
    \item \textbf{Paz, justicia e instituciones sólidas:} promover sociedades pacíficas e inclusivas, proporcionar acceso a la justicia para todos y construir instituciones eficaces, responsables e inclusivas.
    \item \textbf{Alianzas para lograr los objetivos:} fortalecer los medios de ejecución y revitalizar la alianza mundial para el desarrollo sostenible.
\end{enumerate}

Con la realización de este trabajo se comprometen varios puntos de los nombrados anteriormente:
\begin{itemize}
    \item \textbf{Fin de la pobreza:} al realizar este proyecto se generan puestos de trabajos al tener que realizar una administración del sistema y solucionar los problemas de los clientes, por lo que disminuye la tasa de paro.
    \item \textbf{Educación de calidad:} al estar evaluando con diferentes test y cuestionarios se puede cambiar la forma de enseñanza de aquellos que tengan una media baja respecto a lo deseado.
    \item \textbf{Igualdad de género:} con este proyecto no se excluye a nadie, da igual el género que seas o como te sientas en ningún momento se te va a prohibir la realización de cualquier método de evaluación propuesto por RiskReal.
    \item \textbf{Trabajo decente y crecimiento económico:} al estar evaluando las Soft Skills cada persona podrá ser asignada en un puesto que sea más fiel a sí mismo asegurando que tenga un mayor crecimiento, con un fin de que vaya ascendiendo poco a poco.
    \item \textbf{Reducción de las desigualdades:} con el uso de las Soft Skills se puede llegar a estandarizar ciertos puestos de trabajo de varios países con lo que se contribuiría de una manera constante.
\end{itemize}
Una vez tenemos claros varios de los aspectos que se abarcan, podemos llegar a la conclusión de que este proyecto fomenta la utilización de los ODS.