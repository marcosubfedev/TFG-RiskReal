\capitulo{5}{Aspectos relevantes del desarrollo del proyecto}

En este apartado indicare las elecciones que he tomado, el transcurro del proyecto y como se han abordado algún problema relevante.
\section{Elección de las herramientas}
Una vez se empieza a trabajar hay que poner los cimientos, por lo que hay que realizar una selección cuidadosa con los elementos que se va a realizar el proyecto. Por eso con la ayuda de esta \href{ https://stackshare.io/stackups/trending }{ herramienta web} para comparar aspectos de herramientas con el mismo propósito he podido hacer una elección cómoda. 
\subsection{Visual Studio Code}
Para la elección del editor de texto tenemos varias opciones como pueden ser “PyCharm,, Atom, Komodo y VSCodium” todas estas opciones son válidas pero si tenemos que elegir una sería PyCharm la cual no he escogido debido al haber ya trabajado repetidamente con Visual Studio Code y estar más familiarizado, en parte para las necesidades que necesito no hay ninguna extensión que tenga que implementar que no venga ya incluida con Visual Studio Code.
Diferencias entre Visual Studio Code y PyCharm utilizadas para la selección ~\cite{VSC-PyCharm}:
\begin{itemize}
    \item \textbf{Interfaz de usuario:} PyCharm ofrece una interfaz de usuario más completa y con más funciones, que ofrece una experiencia de IDE rica y profesional con herramientas avanzadas de depuración y creación de perfiles. Por otro lado, Visual Studio Code presenta una interfaz de usuario liviana y minimalista, que se centra en la simplicidad y la extensibilidad, con una amplia gama de extensiones disponibles y opciones de personalización.
    \item \textbf{Edición de código:} PyCharm ofrece una potente asistencia de codificación y finalización de código inteligente, brindando sugerencias útiles e importaciones automáticas para módulos de Python. También incluye funciones especializadas para Django, Flask y otros marcos. Si bien Visual Studio Code también brinda finalización de código y sugerencias, es posible que requiera extensiones adicionales para igualar el nivel de inteligencia de código que brinda PyCharm.
    \item \textbf{Herramientas y complementos integrados:} PyCharm incluye un conjunto completo de herramientas integradas para el control de versiones (por ejemplo, Git), marcos de prueba, bases de datos y desarrollo web. Ofrece una integración perfecta con herramientas populares de Python, como entornos virtuales y administradores de paquetes. Visual Studio Code, por otro lado, depende en gran medida del uso de complementos y extensiones para proporcionar funcionalidades similares, lo que permite a los desarrolladores personalizar su entorno con las herramientas que elijan.
    \item \textbf{Rendimiento:} PyCharm está basado en la plataforma IntelliJ, que puede consumir una cantidad significativa de recursos del sistema y generar tiempos de inicio e indexación más lentos en comparación con Visual Studio Code. Visual Studio Code, al ser un IDE liviano, generalmente ofrece un rendimiento más rápido y tiempos de inicio más rápidos.
    \item \textbf{Comunidad y ecosistema:} PyCharm cuenta con una comunidad consolidada y madura con una amplia documentación y foros de soporte. Es ampliamente utilizado en la comunidad de desarrollo de Python, lo que se traduce en una gran cantidad de recursos y tutoriales disponibles. Visual Studio Code, desarrollado por Microsoft, se beneficia de su amplia comunidad y ecosistema, y ofrece una amplia gama de extensiones, temas e integraciones impulsados por la comunidad con otras herramientas de Microsoft.
    \item \textbf{Precios:} PyCharm ofrece ediciones comunitarias tanto gratuitas como de pago. Las versiones de pago ofrecen funciones adicionales y soporte para el desarrollo web y empresarial. Visual Studio Code, por otro lado, es completamente gratuito y de código abierto, lo que lo convierte en una opción atractiva para los desarrolladores que prefieren una solución rentable.
\end{itemize}

\subsection{GitHub}
En cuanto a la selección de la plataforma donde alojar el proyecto GitHub es muy común, tras realizar diferentes proyectos en esta plataforma ya me encuentro familiarizado y no he tenido ninguna intención de realizar el alojamiento en otras plataformas. Por otra parte la integración de la metodología Kanban por parte de Zube resulta más sencillo el manipular las tareas.
Diferencias entre GitHub y GitLab utilizadas para la selección ~\cite{GitHub-GitLab}:
\begin{itemize}
    \item \textbf{Alojamiento e implementación:} GitHub es principalmente una plataforma basada en la nube que ofrece alojamiento para repositorios Git. Proporciona un servicio totalmente administrado, donde los usuarios pueden crear y administrar sus repositorios directamente en los servidores de GitHub. GitLab, por otro lado, ofrece una solución SaaS basada en la nube (GitLab.com) y una solución alojada por los propios usuarios (GitLab Community Edition). Esto significa que los usuarios tienen la flexibilidad de elegir entre alojar sus repositorios en la plataforma en la nube de GitLab o implementar GitLab en su propia infraestructura.
    \item \textbf{Precios y licencias:} GitHub ofrece opciones gratuitas para repositorios públicos y planes pagos para repositorios privados, mientras que GitLab ofrece una Community Edition gratuita y una Enterprise Edition paga.
    \item \textbf{Conjunto de características:} GitHub es conocido por su sólido ecosistema de integración, que incluye una amplia gama de herramientas y servicios de terceros para la integración continua, la implementación y la gestión de proyectos. GitLab, por otro lado, tiene como objetivo proporcionar una plataforma DevOps más completa con características integradas como canalizaciones de CI/CD, integración con Kubernetes y un registro de contenedores Docker integrado.
    \item \textbf{Comunidad y colaboración:} GitHub cuenta con una comunidad de desarrolladores grande y activa, lo que la convierte en una plataforma popular para proyectos de código abierto. Ofrece funciones que facilitan la colaboración y la contribución, como solicitudes de incorporación de cambios, revisiones de código y seguimiento de problemas. GitLab también cuenta con una comunidad en crecimiento y su naturaleza de código abierto fomenta la colaboración. Además, GitLab enfatiza el concepto de "Flujo de GitLab", que promueve un flujo de trabajo más integrado y optimizado para el desarrollo, la colaboración y la implementación. Esto incluye funciones como canales de CI/CD integrados y una interfaz consolidada para administrar el código, los problemas y la gestión de proyectos.
\end{itemize}

\subsection{Flask}
Para la realización de la página web una manera sencilla de empezar en el caso de no haber realizado ninguna anteriormente, es Flask por lo cual entre la propia ayuda que te ofrece la documentación de flask y el gran abanico de librerías que ofrece son suficientes para el proyecto que tenía en mente.
Diferencias entre GitHub y GitLab utilizadas para la selección ~\cite{Flask-Django}:
\begin{itemize}
    \item \textbf{Tipo de framework:} Django es un framework web de alto nivel y de pila completa que sigue el patrón de diseño Modelo-Vista-Plantilla (MVT), y ofrece funciones integradas como autenticación, interfaz de administración y ORM. Por otro lado, Flask es un framework web ligero que se basa en Werkzeug y Jinja2 y ofrece flexibilidad a los desarrolladores para elegir las herramientas y bibliotecas con las que quieren trabajar.
    \item \textbf{Escalabilidad:} Django es más adecuado para aplicaciones más grandes debido a sus características y convenciones integradas que ayudan a gestionar proyectos complejos de manera eficiente. Flask, al ser más minimalista, es más adecuado para proyectos más pequeños o aplicaciones que requieren un enfoque más personalizado.
    \item \textbf{Soporte y ecosistema de la comunidad:} Django tiene una comunidad más grande y madura en comparación con Flask, y ofrece una amplia gama de paquetes de terceros, tutoriales y documentación que pueden ayudar a los desarrolladores a crear e implementar aplicaciones más rápido. Flask, por otro lado, tiene una comunidad más pequeña, pero ofrece un ecosistema más liviano y flexible que permite a los desarrolladores tener más control sobre sus proyectos.
    \item \textbf{Curva de aprendizaje:} Django tiene una curva de aprendizaje más pronunciada en comparación con Flask debido a sus características integrales y convenciones que los desarrolladores deben comprender y seguir. Flask, al ser más simple y minimalista, tiene una barrera de entrada más baja, lo que facilita que los principiantes comiencen a crear aplicaciones web.
    \item \textbf{Enrutamiento de URL:} En Django, el enrutamiento de URL se realiza a través de una configuración de URL predefinida mediante expresiones regulares, lo que lo hace más estructurado y organizado. Flask, por otro lado, utiliza decoradores para definir rutas, lo que proporciona una forma más flexible e intuitiva de enrutar URL en la aplicación.
    \item \textbf{Motor de plantillas:} Django viene con un motor de plantillas integrado que sigue los principios DRY (Don't Repeat Yourself), lo que permite a los desarrolladores reutilizar el código de las plantillas de manera eficiente. Flask, por otro lado, utiliza Jinja2 como su motor de plantillas predeterminado, lo que proporciona una sintaxis más liviana y fácil de usar para el diseñador para crear plantillas.
\end{itemize}

\section{Desarrollo del proyecto}
Para el desarrollo he seguido unas pautas que han sido claves, las cuales consisten en:
\begin{enumerate}
    \item Plantear la idea a fondo con un propósito adicional al deseado para abarcar posibles cambios.
    \item Realizar un primer código con el cual se logre lo deseado y después adaptarlo para obtener la funcionalidad deseada.
    \item Realizar una prueba de todo el programa para depurar y comprobar que no haya ningún error.
\end{enumerate}

\section{Problemas}
Los problemas  siempre aparecen y como no podía ser durante el desarrollo de la aplicación surgieron diferentes errores los cuales se han ido desarrollado. 

Un problema que tuve fue la visualización de caracteres erróneos cuando el texto disponía de alguna tilde o símbolo de interrogación, al estar trabajando con todos los ficheros en “UTF-8” comprobándolo con el propio “Visual Studio Code” o el “notepad++” y de asegurar que el texto del html estuviese codificado también adecuadamente,  no tenía el porqué del error, pero tras darle algunas vueltas se me ocurrió la idea de que no estuviese tratando el fichero correctamente y de esta manera lo solucione al especificar como quería abrir el fichero en cuestión para obtener los datos ya así codificados en “UTF-8”. 

Durante la implementación del usuario invitado para poder realizar la evaluación sin estar registrado, me encontré con que en un principio al solo tener la evaluación de un test al añadir el modelo de los cuestionarios tuve el dilema de cómo poner en común al invitado, por lo que para poder solucionarlo cree una función intermedia en la cual comprobaba si estaba logeado algún usuario en la sesión así pudiendo  utilizar el mismo invitado en común para luego llamar a la función correspondiente dependiendo si se quiere hacer un test o cuestionario. 

A la hora de utilizar Docker para crear un contenedor de la aplicación me surgió un error en el cual  las imágenes correspondientes al test no se muestran dando como fallo que no se encuentran, el problema en cuestión resultaba en que las imágenes no se encontraban en el destino pero viendo los archivos del contenedor las imágenes si se encontraban, lo cual para solucionarlo probé diferentes maneras de llegar a la ruta pero al final el error en si es que los propios archivos tenían la extensión  \textbf{.PNG} mientras utilizaba la extensión \textbf{.png} como ruta para cargar las imágenes, como docker genera una máquina virtual con un sistema operativo de Linux, al tratar los archivos las extensiones son importantes, por lo que para solucionarlo tuve que cambiar la extensión de la ruta que apunta a las imágenes del json a \textbf{.PNG}.