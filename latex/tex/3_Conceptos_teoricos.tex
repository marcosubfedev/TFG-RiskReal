\capitulo{3}{Conceptos teóricos}

En este apartado se definen algunos términos y conceptos relevantes sobre el proyecto.

\subsection{Soft Skills}
Las habilidades blandas ~\cite{wiki:SoftSkills}, competencias blandas o habilidades suaves  son una combinación de habilidades sociales, habilidades de comunicación, rasgos de la personalidad, actitudes, atributos profesionales, inteligencia social e inteligencia emocional, que facultan a las personas para moverse por su entorno, trabajar bien con otros, realizar un buen desempeño y, complementándose con las habilidades duras, conseguir sus objetivos.
Las habilidades blandas son un cúmulo de rasgos productivos de la personalidad que caracterizan las relaciones de una persona en un medio. Estas habilidades pueden incluir autoestima, comunicación, elocuencia, hábitos personales, empatía, gestión del tiempo, trabajo en equipo y liderazgo. Una definición basada en la revisión de artículos al respecto considera "habilidades blandas" un término paraguas para habilidades con tres elementos funcionales clave: interpersonales, sociales y profesionales.
Las habilidades blandas se pueden categorizar en diferentes como estas:
\begin{enumerate}
    \item \textbf{Comunicación}: capacidad de expresarse claramente hablando y escribiendo, y de hacer presentaciones en público.
    \item \textbf{Cortesía}: buenos modales, etiqueta, respeto, decir «por favor» y «gracias».
    \item \textbf{Flexibilidad}: adaptabilidad, disposición al cambio y a la formación continua, aceptación de lo nuevo, se le puede enseñar.
    \item \textbf{Integridad}: sinceridad, moralidad, valores personales, honradez.
    \item \textbf{Habilidades interpersonales}: agradable, con sentido del humor, amistoso, hospitalario, empático, con autocontrol, paciencia, sociabilidad, calidez, habilidades sociales.
    \item \textbf{Actitud positiva}: optimista, entusiasta, ánima a sus compañeros, feliz, seguro.
    \item \textbf{Profesionalidad}: bien vestido, buen aspecto, compostura.
    \item \textbf{Responsabilidad}: responsable, fiable, termina el trabajo, quiere hacerlo bien, es consciente, tiene recursos, autodisciplina y sentido común.
    \item \textbf{Trabajo en equipo}: coopera, se lleva bien con otros, es agradable, anima, ayuda a quien lo necesita.
    \item\textbf{Actitud hacia el trabajo}: trabaja duro, está disponible, es leal, puntual, tiene iniciativa, dedicación, motivación y falta poco al trabajo.
    \item \textbf{Creatividad} ~\cite{wiki:Creatividad}: capacidad de crear nuevas ideas o conceptos, de nuevas asociaciones entre ideas y conceptos conocidos, que habitualmente producen soluciones originales.
    \item \textbf{Resolución de problemas} ~\cite{wiki:ResProblemas}: la conclusión de un proceso más amplio que tiene como pasos previos la identificación del problema y su modelado.
    \item \textbf{Ser organizado}: llevar a cabo unos pasos de manera seguida non un propósito aprovechando el tiempo lo mas posible.
\end{enumerate}
Estas pueden ser utilizadas para poder hacer un cribado de un gran grupo de personas y así poder realizar una agrupación con diferentes integrantes que tengas diferentes atributos como el de líder, trabajador, creativo, espontaneo, observador, etc. ; o por otra parte hacer una selección para un nuevo proyecto a partir de una semejanza a los conocimientos deseados.
~\cite{softskillsacademy}
\imagen{softSkills}{Tabla de Soft Skills}{1}