\capitulo{7}{Conclusiones y Líneas de trabajo futuras}

Para finalizar se expondrán las conclusiones finales obtenidas a la realización del proyecto y por otra parte las líneas de trabajo futuras.

\subsection{Conclusiones}
A medida que se avanzaba en el proyecto los objetivos marcados se realizaban satisfactoriamente en su mayoría aun al no haber podido realizar alguna funcionalidad de la manera deseada.
La navegación entre las pestañas se hace intuitiva y no es pesado lo cual nos proporciona una página web que es accesible y no necesita tener una guía a lado para poder interactuar con ella satisfactoriamente.
Al realizar el proyecto se ha obtenido una perspectiva que no se tenía respecto a cómo desarrollar una página web y diferentes funcionalidades que debería tener de base.
Como resumen la realización de este proyecto pone a prueba la utilización de diferentes aspectos enseñados en la carrera, además que se hace un planteamiento único que resuelve los diferentes problemas que van surgiendo a medida. 
\subsection{Líneas de trabajo futuras} 
Aunque se tenga una funcionalidad correcta y unas funcionalidades satisfechas, siempre es posible realizar mejoras para tener un producto mejor.
\begin{itemize}
    \item Mejora de la lógica de cambio de contraseña aplicando una doble verificación al enviar un mensaje al correo asociado a la cuenta que desea cambiar la contraseña.
    \item Implementación de una opción que te permita modificar los cuestionarios/tests o crear nuevos desde la propia aplicación web.
    \item Implementación de un nuevo rol “editor” que sea el encargado del apartado anterior.
    \item Poder hacer un filtrado sobre los diferentes campos respecto a las tablas de usuarios y los resultados.
    \item Creación de unas pruebas unitarias para comprobar un funcionamiento correcto sin tener que realizarlo a mano.
\end{itemize}
